\documentclass{article}

\usepackage[utf8]{inputenc}
\usepackage{unicode-helper}
\usepackage{amsmath}
\usepackage{amssymb}
\usepackage{amsfonts}
\usepackage{microtype}
\usepackage{tikz-cd}
\usepackage[english]{babel}
\usepackage{amsthm}
\usepackage{mathtools}
\usepackage{braket}
\usepackage{csquotes}
\usepackage[backend=biber,style=numeric]{biblatex}
\usepackage[pdftex]{hyperref}
\usepackage{cleveref}

\newcommand{\titlevar}{Notes about Prime Constellations}
\newcommand{\authorvar}{Günthner}
\newcommand{\datevar}{Winter 2024}
\title{\titlevar}
\author{\authorvar}
\date{\datevar}
\hypersetup{
	pdftitle=\titlevar,
	pdfauthor=\authorvar,
	pdfcreationdate=\datevar,
}
\setlength{\parindent}{0pt}

\newlength{\halflength}
\setlength{\halflength}{100pt}

\newcommand{\card}[1]{\#{#1}}
\newcommand{\inv}{^{-1}}
\newcommand{\ringunits}[1]{{#1}^{\mathclap{\rule{2.5pt}{0pt}\times}\rule{0.2em}{0pt}}}
\newcommand{\ringunitsb}[1]{\ringunits{\left({#1}{\vphantom{p^k}}\right)}}
\newcommand{\ordgroup}[1]{\ord_{\rule{0pt}{9pt}\mathclap{#1}}}
\newcommand{\ordadd}[1]{\ordgroup{ℤ/{#1}ℤ}}
\newcommand{\ordmult}[1]{\ord_{\rule{0pt}{8pt}{
			\settowidth{\halflength}{$\scriptstyle \left({ℤ/{#1}ℤ}{\vphantom{p^k}}\right)$}
			\mathrlap{\rule{-0.5 \halflength}{0pt}\ringunitsb{ℤ/{#1}ℤ}}
}}}
\newcommand{\ordker}[1]{\ordgroup{\ker(\mathrlap{#1)}}}
\newcommand{\frakB}{\mathfrak{B}}
\newcommand{\frakc}{\mathfrak{c}}
\newcommand{\bigbarn}[1]{\big({#1}\big)}
\newcommand{\Bigbarn}[1]{\Big({#1}\Big)}
\newcommand{\spospart}[1]{ {(#1)} _ {\rule{1.5pt}{0pt}\mathclap +} }
\newcommand{\pospart}[1]{{ {\bigbarn{#1}} _ {\mathclap +} }}
\newcommand{\tritem}{\item[$\blacktriangleright$]}
\newcommand{\gspan}[1]{<#1>}
\DeclareMathOperator{\ordb}{ord}
\newcommand{\ord}{\mathop{\ordb}\limits}
\newcommand{\powerset}{\mathcal{P}}
\DeclareMathOperator{\lcm}{lcm}
\DeclareMathOperator{\id}{id}

\newcommand{\maybeequal}{\stackrel{?}{=}}
\newcommand{\mapdefinition}[5]{
	\begin{center}
		\begin{tabular}{llll}
			$#1:$ 	&	$#2$ & $→$ & $#3$ 	\\
			&	$#4$ & $↦$ & $#5$	\\
		\end{tabular}
	\end{center}
}
\newcommand{\padlr}[2]{\rule{#2}{0pt} {#1} \rule{#2}{0pt}}
\newcommand{\buildset}[2]{\set{{#1} \ : \ {#2}}}
\newcommand{\floor}[1]{\left\lfloor {#1} \right\rfloor}
\newcommand{\ceil}[1]{\left\lceil {#1} \right\rceil}

\newtheorem{definition}{Definition}
\newtheorem{lemma}{Lemma}
\newtheorem{theorem}{Theorem}
\newtheorem{conjecture}{Conjecture}

\crefname{lemma}{Lemma}{Lemmata}
\crefname{equation}{equation}{equations}
\crefname{theorem}{Theorem}{Theorems}
\crefname{definition}{Definition}{Definitions}

%\addbibresource{Order Formula.bib}

\begin{document}
	\maketitle
	
	\section{Introduction}
	
	\begin{definition}
		A Constellation is a function: $χ: ℝ → ℝ^d$ \\
		We call $d$ the degree of the constellation.
	\end{definition}
	
	We are now looking for $n$ such that all $χ(n)$ are prime! Or rather we are looking for their count in the range $[1,n]$. To do this we will first \enquote{generalize} prime numbers because we have to closed form expression for $π(n)$, the number of primes in the range $[1,n]$. Instead we have an upper bound.

	\medskip
	
	The general idea is to step over all \enquote{prime} numbers and eliminate all $t$ such that not all $χ(t)$ are prime. Then we use the upper bound to establish a lower bound for the resulting count.
	
	\subsection{Example Constellations}
	
	\subsubsection{Basic Prime Numbers}
	
	The constellation $χ(t)= t$ yields the primes.
	
	\subsubsection{Twin Primes}
	
	The constellation $χ(t)= (t-2, t)$ yields twin primes.
	
	\subsubsection{Goldbach-n-primes}
	
	\begin{definition} Goldbach-$n$-prime \\
		For $n$ even, a number $p<n$ is a Goldbach-$n$-prime iff both $p$ and $n-p$ are prime
	\end{definition}
	\begin{conjecture}[The Goldbach-Conjecture]
		For every $n≥4$ there exists at least one Goldbach-$n$-prime
	\end{conjecture}
	
	\medskip
	
	The constellation $χ(t)= (t, n-t)$ yields Goldbach-$n$-primes
	
	\subsubsection{Landau's Fourth Problem}
	
	The constellation $χ(t)= t²+1$ yields primes for Landau's fourth problem.
	
	\subsubsection{Mersenne Primes}
	
	The constellation $χ(t) = 2^t-1$ yields Mersenne Primes.
	
	\subsubsection{Sophie Germain Primes}
	
	The constellation $χ(t) = (t, 2t+1)$ yields Sophie Germain Primes.
	
	\section{Basic Definitions}
	
	Let's define a function that generalizes prime numbers:
	\begin{equation}
		ψ: ℤ → ℤ
	\end{equation}
	
	We will also need an inverse function for $χ$:
	\begin{equation}
		\begin{split}
			χ\inv : ℝ &→ ℝ^d \\
			χ\inv(t) &= (χ\inv_k(t))_{kϵ[d]}
		\end{split}
	\end{equation}
	
	Now to define the numbers \enquote{coprime} to the first $k$ \enquote{prime} numbers: $M_χ^ψ(k)$
	\begin{equation}
		\begin{split}
			M_χ^ψ(0) \coloneq M_0 \coloneq&\ ℤ \setminus \buildset{tϵℤ}{∃ lϵ[d]: χ_l(t)=0 ∨ χ_l(t)ϵ\ringunits{ℤ}} \\
			=&\ ℤ \setminus χ\inv(0) \setminus χ\inv(\ringunits{ℤ})
		\end{split}
	\end{equation}
	\begin{equation}
		\begin{split}
			M_χ^ψ(k) \coloneq&\ M_χ^ψ(k-1) \setminus \buildset{tϵℤ}{∃ lϵχ(t): ψ(k) \mid l} \\
			=&\ M_χ^ψ(k-1) \setminus \buildset{tϵℤ}{∃ lϵ[d], mϵM_0 : mψ(k) = χ_l(t)}\\
			=&\ M_χ^ψ(k-1) \setminus \buildset{tϵℤ}{∃ lϵ[d], mϵM_0 : χ\inv_l(mψ(k)) = t}\\
			=&\ M_χ^ψ(k-1) \setminus χ\inv\left(ψ(k) · M_0\right) \\
			=&\ M_χ^ψ(k-1) \setminus \Big( M_χ^ψ(k-1) \cap χ\inv\left(ψ(k) · M_0\right) \Big) \\
		\end{split}
	\end{equation}
	
	\begin{definition}
		\begin{equation}
			Ψ_χ^ψ = \lim_{k→∞} M_χ^ψ(k)
		\end{equation}
	\end{definition}
	
	\begin{definition}
		\begin{equation}
			π_χ^ψ(t) = \card{(Ψ_χ^ψ \cap [t])}
		\end{equation}
	\end{definition}
	
	\section{Derivation}
	
	\subsection{Primes}
	
	For $χ=\id$:
	\begin{equation}
		\begin{split}
			M_χ^ψ(k-1) \cap χ\inv\left(ψ(k) · M_0\right) &= M_χ^ψ(k-1) \cap \left(ψ(k) · M_0\right) \\
			&= ψ(k) · M_χ^ψ(k-1)
		\end{split}
	\end{equation}
	
	\begin{equation}
		π_{\id}^ψ(t) = γ_{ψ\inv(t)}^ψ(t)
	\end{equation}
	\begin{equation}
		γ_k^ψ(t) \coloneq M_χ^ψ(t) \cap [t]
	\end{equation}
	\begin{equation}
		γ_0^ψ(t) = t
	\end{equation}
	\begin{equation}
		\begin{split}
			γ_k^ψ(t) = γ_{k-1}^ψ(t) - γ_{k-1}^ψ(\floor{\frac{t}{ψ(t)}})
		\end{split}
	\end{equation}
	
	\subsection{Twin Primes}
	
	For $χ(t)=\set{t-2, t}$:
	\begin{equation}
		\begin{split}
			M&_χ^ψ(k-1) \cap χ\inv\left(ψ(k) · M_0\right)\\
			 &= M_χ^ψ(k-1) \cap (ψ(k) · M_0 \cup (ψ(k) · M_0 + 2)) \\
			&= (M_χ^ψ(k-1) \cap ψ(k) M_0) \ \cup\ (M_χ^ψ(k-1) \cap ψ(k)M_0 + 2)
		\end{split}
	\end{equation}
	
	\subsection{Goldbach-n-primes}
	
	For $nϵℕ$ and $χ(t)=\set{t, n-t}$:
	\begin{equation}
		\begin{split}
			M&_χ^ψ(k-1) \cap χ\inv\left(ψ(k) · M_0\right) \\
			&= (M_χ^ψ(k-1) \cap ψ(k)M_0) \cup (M_χ^ψ(k-1) \cap (n - ψ(k)M_0))
		\end{split}
	\end{equation}
	
	\subsection{Landau's Fourth Problem}
	
	\begin{equation}
		\begin{split}
			M_χ^ψ(k-1) \cap χ\inv\left( ψ(K)M_0 \right) &= M_χ^ψ(k-1) \cap \sqrt{ψ(k)M_0 -1}
		\end{split}
	\end{equation}
	
	\subsection{Mersenne Primes}
	
	\begin{equation}
		\begin{split}
			M_χ^ψ(k-1) \cap χ\inv\left( ψ(K)M_0 \right) &= M_χ^ψ(k-1) \cap \log_2(ψ(k)M_0 + 1)
		\end{split}
	\end{equation}
	
	\subsection{Sophie Germain Primes}
	
	\begin{equation}
		\begin{split}
			M&_χ^ψ(k-1) \cap χ\inv\left( ψ(K)M_0 \right) \\
			&= (M_χ^ψ(k-1) \cap ψ(k)M_0) \cup (M_χ^ψ(k-1) \cap \frac{ψ(k)M_0 - 1}{2})
		\end{split}
	\end{equation}
	
	\section{Results}

	
	\begin{lemma}
		\begin{equation}
			\mathbb{P} = Ψ_{\id}^{π\inv}
		\end{equation}
	\end{lemma}
	
	\begin{lemma}
		\begin{equation}
			π(t) = π_{\id}^{π\inv}(t)
		\end{equation}
	\end{lemma}
	
	\section{Notes}
	
	\begin{tabular}{|c|c|c|c|c|c|c|c|c|c|}
		\hline
		$2$ & $3$ & & $5$ & & $7$ & & & & \\
		\hline
		& & $2$ & $3$ & & $5$ & & $7$ & &  \\
		\hline
	\end{tabular}
\end{document}